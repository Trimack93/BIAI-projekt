% !TeX spellcheck = pl_PL
\documentclass[a4paper,twoside]{article}
\usepackage{polski}
\usepackage[utf8]{inputenc}
\usepackage[pdftex]{graphicx}
\usepackage{amsmath}

\usepackage[unicode, bookmarks=true]{hyperref} %do zakładek
\usepackage{tabto} % do tabulacji
\NumTabs{11} % globalne ustawienie wielkosci tabulacji
\usepackage{array}
\usepackage{multirow}
\usepackage{array}
\usepackage{dcolumn}
\usepackage{bigstrut}
\usepackage{color}
\usepackage[usenames,dvipsnames]{xcolor}
\usepackage{pdfpages}
\usepackage{sidecap}
\usepackage{wrapfig}
\usepackage{float}	%for figure& table placement in text
\usepackage{textgreek}


\setlength{\textheight}{24cm}
\setlength{\textwidth}{15.92cm}
\setlength{\footskip}{10mm}
\setlength{\oddsidemargin}{0mm}
\setlength{\evensidemargin}{0mm}
\setlength{\topmargin}{0mm}
\setlength{\headsep}{5mm}
\newcommand{\HRule}{\rule{\linewidth}{0.5mm}} 

\newcolumntype{M}[1]{>{\centering\arraybackslash}m{#1}}
\newcolumntype{N}{@{}m{0pt}@{}}

\graphicspath{ {./img/} }

% === Reset inkrementacji sekcji przy nowym parcie === %
\usepackage{titlesec}

\makeatletter
\@addtoreset{section}{part}
\makeatother
\titleformat{\part}[display]
{\normalfont\LARGE\bfseries\left}{}{0pt}{}

\titleformat{\chapter}[hang]{\LARGE\bfseries}{\thechapter\hsp\textcolor{blue}{|}\hsp}{0pt}{\Huge\bfseries}


\begin{document}
	
	\begin{titlepage}
		\begin{center}
			
			% Upper part of the page. The '~' is needed because \\
			% only works if a paragraph has started.
			\includegraphics[width=0.5\textwidth]{./img/logo.png}~\\[1cm]
			%?[width=0.15\textwidth]
			
			\textsc{\LARGE Politechnika Śląska w Gliwicach}\\[1.5cm]
			
			\textsc{\LARGE Biologically Inspired Artificial Intelligence}\\[0.2cm]
			
			\textsc{\LARGE Raport z projektu}\\[0.2cm]
			
			% Title
			\HRule \\[0.4cm]
			{ \Huge \bfseries Predykcja jakości obrazów na podstawie metadanych \\[0.4cm] }
			
			\HRule \\[1.5cm]
			
			% Author and supervisor
			\textsc{\Large Autorzy:} \\
			Bartłomiej Buchała \\
			Marek Motyka \\[1.0cm]
			
			Informatyka, semestr VI \\
			Rok akademicki 2014/2015 \\
			Grupa GKiO3
			
			\vfill
			
			% Bottom of the page
			{\large \today}
			
		\end{center}
	\end{titlepage}
	
\newpage

\section{Temat projektu}

\section{Implementacja algorytmu}

\section{Wykorzystane biblioteki}

\section{Podsumowanie}

\end{document}